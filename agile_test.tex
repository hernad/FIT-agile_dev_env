\documentclass[times, utf8, seminar]{fit}

\usepackage{listings}
\usepackage{longtable}
\usepackage{xcolor}
\usepackage{float}
\usepackage{enumitem}
\usepackage{hyperref}
\usepackage{enumerate}
\usepackage{graphicx}
\usepackage{etoolbox}
\usepackage{datetime}
\usepackage{needspace}
\usepackage{titlesec}

\begin{document}
\widowpenalty=300
\clubpenalty=300
\setlength{\parindent}{0pt}

\lstset{
  language=bash,
  backgroundcolor=\color{gray!25},
  basicstyle=\ttfamily \footnotesize,
  breaklines=true,
  prebreak=\raisebox{0ex}[0ex][0ex] {\ensuremath{\hookleftarrow}},
  columns=fullflexible,
  keywords={},
  mathescape=false
}

\title{Agilni \emph{software development}\newline Tehnike testiranja}

\author{Ernad Husremović}
\brindex{DL 2792}
\verzija {0.0.3}
\mentor{mr. Adil Joldić}

\maketitle

\tableofcontents

%\listoftables
%\listoffigures
\newpage

% abstract begin
%\begin{abstract}
%
%To be done 
%
%\keywords{open source software, OSS, Bosna i Hercegovina}
%\end{abstract}

% abstract end

\chapter{Uvod}
\vspace*{-0.7cm}

Testiranje je neizostavna tehnika agilnog software developmenta.

\chapter{\emph{Test driven development} (TDD)}

TDD se često uzima za "ekstremnu" agilnu tehniku. Naime, TDD kod implementacije kao prvi korak stavlja pisanje testa, a ne pisanje produkcijskog k\^oda - ''test first''. Jedan TDD ciklus se dijeli na sljedeće faze:

\begin{enumerate}
  \item Misli
  \item Crveno
  \item Zeleno
  \item refactor
  \item Ponovo ...
\end{enumerate}

\section{TDD i \emp{Legacy} k\^od}

Jedna od definicija Legacy k\^oda glasi ovako:
\begin{quote}
  \emph{Legacy k\^od je k\^od bez testova}\footnote{\citep[str. 303]{agileart}}
\end{quote}

Pogledati takođe i sekciju ''Legacy system i CI''\citep[str. 3]{agileci}

\section{Unit tests}

Unit testovi se fokusiraju na određenu klasu ili metod. Oni se trebaju izvršiti u potpunosti u memoriji.\citep[str. 300]{agileart}

\section{Refactoring}

Reflektivni dizajn\citep[str. 306]{agileart}

Detekcija \emph{Code smells}
\begin{itemize}
  \item Divergent Change - klasa sadrži previše koncepata
  \item Shotgun Surghery - suprotno od predhodnog, više klasa obrađuje jedan koncept
  \item Primitive Obsession - konceptvi višeg nivoa prezentovani primitivnim tipovima (Decimal prezentuje Money klasu)
  \item Data Clumps - slično predhodnom - više primitivnih koncepata je grupirano, umjesto da bude enkapsulirano u jednu klasu\footnote{indikator je kada istu grupu varijabli prosljeđujemo između metoda}
  \item Data Class and Wannabee Static Class - data i k\^od u različitim klasama.
  \item Coding Nulls - umjesto kodiranja NULL vrijednosti primjeniti ''fast fail'' strategiju
  \item Time Depenedencies - Umjesto da svojim upravlja stanjem, klasa očekuje da pozivaoci postave njeno stanje
  \item Half baked objects - posebna varijanta predhodnog slučaja - ovakvi objekti moraju biti prvo konstruirani, pa inicijalizirani nekom metodom, pa tek na kraju korišteni 
  
\end{itemize}

\subsection{Analiza k\^oda}

Šta je svrha određenog komponente (paketa, klase) ?

Koje interakcije ostvaruje sa drugim komponenntama ?

Alati za analizu:
\begin{itemize}
  \item UML sekvencijalni dijagrami
  \item CRC kartice
\end{itemize}


\subsection{Refactoring != rewriting}

Refactoring je niz \emph{malih} trasformacija k\^oda.
Svaka značajna promjena dizajna traži niz \emph{refactoring}-a.

Refactoring se obavlja konstantno i svakodnevno. Princip ''malih koraka'' nam omogućava da refactoring ne remeti realizaciju našeg glavnog zadatka.

\section{Integracijski testovi}

\subsection{Fokusirani integracijski testovi}

Testiranje kod koga se unutar testova dešava interakcija sa fajl sistemom, mrežnim resursima.

\subsection{End-To-End integracijski testovi}

Testiranje korisničkog interfejsa

\section{Brzina izvršenja testova}

Veoma bitan faktor testova je brzina izvršenja. Ako je za izvođenje testova potrebno više od 1-2 sekunde, oni se moraju isključiti iz TDD procesa. 

''Spori'' integracijski testovi bi se pokretali jednom ili par puta u toku dana manuelno, odnosno automatski tokom CI procesa\footnote{\citep{agileci}}.

\section{Testiranje i programski jezici}



\section{Type safe jezici}

C++ Java imaju kompilacijusku fazu koja provjerava sintaksne greške.

\section{Dinamički jezici}

\section{Testiranje vs debugiranje}

Više testiranja, manje potrebe za debugiranjem.


\chapter{Customer tests}

\citep[str. 308]{agileart}



\chapter{\emph{Exploratory Testing}}

Istraživačko testiranje (eng. Exploratory Testing) je glavni zadatak testera. Istraživačko testiranje sadrži ljudski element koji automatske metode testiranja (\emph{unit tests}, \{integration tests}) testiranja ne mogu nadomjestiti.\citep[str. 342]{agileart}

Istraživačka testna sesija traje 1 do 2 sahata. Testeri se tokom sesije služe sa sljedeće četiri tehnike - alata:
\begin{enumerate}
  \item \emph{Charters} - (\href{http://translate.google.com/#en/hr/spike}{\color{blue}{bos. čarter, iznajmljivanje}})
  \item Obzervacija
  \item Uzimanje bilješki
  \item Heuristika
\end{enumerate}


\chapter{Test \emph{frameworks}}

\section{node.js testing frameworkds}

\subsection{Jasmine}

Browser, node.js

%\subsection{vows}

%\subsection{mocha}

%\url{http://visionmedia.github.com/mocha}


%web scrapping

%\url{http://net.tutsplus.com/tutorials/javascript-ajax/web-scraping-with-node-js}

%http://www.akawebdesign.com/2012/01/23/automating-javascript-unit-tests-with-git/

%pre-commit phantom-js:

%\url{https://github.com/arthurakay/Prize-Patrol/tree/master/test}


%\url{http://sinonjs.org}

%\url{https://github.com/vojtajina/testacular}


\chapter{Zaključak}

TODO.

% -------------------------------------------------
\bibliography{literatura}
\bibliographystyle{fit}

\end{document}
