\documentclass[times, utf8, seminar]{fit}

\usepackage{listings}
\usepackage{longtable}
\usepackage{xcolor}
\usepackage{float}
\usepackage{enumitem}
\usepackage{hyperref}
\usepackage{enumerate}
\usepackage{graphicx}
\usepackage{etoolbox}
\usepackage{datetime}
\usepackage{needspace}
\usepackage{titlesec}

\begin{document}
\widowpenalty=300
\clubpenalty=300
\setlength{\parindent}{0pt}

\lstset{
  language=bash,
  backgroundcolor=\color{gray!25},
  basicstyle=\ttfamily \footnotesize,
  breaklines=true,
  prebreak=\raisebox{0ex}[0ex][0ex] {\ensuremath{\hookleftarrow}},
  columns=fullflexible,
  keywords={},
  mathescape=false
}

\title{Agilni \emph{software development}\newline Tehnike testiranja}

\author{Ernad Husremović}
\brindex{DL 2792}
\verzija {0.0.1}
\mentor{mr. Adil Joldić}

\maketitle

\tableofcontents

%\listoftables
%\listoffigures
\newpage

% abstract begin
%\begin{abstract}
%
%To be done 
%
%\keywords{open source software, OSS, Bosna i Hercegovina}
%\end{abstract}

% abstract end

\chapter{Uvod}
\vspace*{-0.7cm}

Testiranje je neizostavna tehnika agilnog software developmenta.

\chapter{Test driven development (TDD)}

\section{Unit tests}

Unit testovi se fokusiraju na određenu klasu ili metod. Oni se trebaju izvršiti u potpunosti u memoriji.\citep[str. 300]{agileart}

\section{Integracijski testovi}

\subsection{Fokusirani integracijski testovi}

Testiranje kod koga se unutar testova dešava interakcija sa fajl sistemom, mrežnim resursima.

\subsection{End-To-End integracijski testovi}

Testiranje korisničkog interfejsa

Zašto je gornja podjela testova bitna ? Veoma bitan faktor testova je brzina izvršenja. Ako je za izvođenje testova potrebno više od 1-2 sekunde, oni se moraju isključiti iz TDD procesa.

\chapter{Istraživačko testiranje}

Istraživačko testiranje (eng. Exploratory Testing) je glavni zadatak testera. Istraživačko testiranje sadrži ljudski element koji automatske metode testiranja (\emph{unit tests}, \{integration tests}) testiranja ne mogu nadomjestiti.\citep[str. 342]{agileart}

Istraživačka testna sesija traje 1 do 2 sahata. Testeri se tokom sesije služe sa sljedeće četiri tehnike - alata:
\begin{enumerate}
  \item \emph{Charters} - (\href{http://translate.google.com/#en/hr/spike}{\color{blue}{bos. čarter, iznajmljivanje}})
  \item Obzervacija
  \item Uzimanje bilješki
  \item Heuristika
\end{enumerate}

\chapter{Zaključak}

TODO.

% -------------------------------------------------
\bibliography{literatura}
\bibliographystyle{fit}

\end{document}
