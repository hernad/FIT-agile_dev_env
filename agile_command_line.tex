\documentclass[times, utf8, seminar]{fit}

\usepackage{listings}
\usepackage{longtable}
\usepackage{xcolor}
\usepackage{float}
\usepackage{enumitem}
\usepackage{hyperref}
\usepackage{enumerate}
\usepackage{graphicx}
\usepackage{etoolbox}
\usepackage{datetime}
\usepackage{needspace}
\usepackage{titlesec}

\begin{document}
\widowpenalty=300
\clubpenalty=300
\setlength{\parindent}{0pt}

\lstset{
  language=bash,
  backgroundcolor=\color{gray!25},
  basicstyle=\ttfamily \footnotesize,
  breaklines=true,
  prebreak=\raisebox{0ex}[0ex][0ex] {\ensuremath{\hookleftarrow}},
  columns=fullflexible,
  keywords={},
  mathescape=false
}

\title{Agilni \emph{software development}\newline \emph{Zašto developeri vole ''konzolu''}}


% http://en.wikipedia.org/wiki/System_console
% http://en.wikipedia.org/wiki/Computer_terminal

% http://tldp.org/HOWTO/Text-Terminal-HOWTO-5.html#ss5.1

% http://en.wikipedia.org/wiki/Terminal_emulator


% A terminal emulator, terminal application, term, or tty for short, is a program that emulates a video terminal within some other display architecture. Though typically synonymous with a command line shell or text terminal, the term terminal covers all remote terminals, including graphical interfaces. A terminal emulator inside a graphical user interface is often called a terminal window.


% programerski kalkulator

% http://stackoverflow.com/questions/84421/converting-an-integer-to-a-hexadecimal-string-in-ruby
% irb> "%x" % (0x0FF + 0x0EE) 
% => "1ed" 


\author{Ernad Husremović}
\brindex{DL 2792}
\verzija {0.1.0}
\mentor{mr. Adil Joldić}

\maketitle

\tableofcontents

%\listoftables
%\listoffigures
\newpage

% abstract begin
%\begin{abstract}
%
%To be done 
%
%\keywords{open source software, OSS, Bosna i Hercegovina}
%\end{abstract}

% abstract end

\chapter{Uvod}

%pipe ls | vako | nako


% http://lifehacker.com/385929/best-text-editors

% http://sourceforge.net/apps/mediawiki/notepad-plus/index.php?title=Plugin_Central


% vim
% https://github.com/joonty/vdebug
% https://github.com/thoughtbot/vimulator


% http://en.wikipedia.org/wiki/Human_computer_interaction

% Human-omputer Interaction (HCI) involves the study, planning, and design of the interaction between people (users) and computers. It is often regarded as the intersection of computer science, behavioral sciences, design and several other fields of study. The term was popularized by Card, Moran, and Newell in their seminal 1983 book, "The Psychology of Human-Computer Interaction", although the authors first used the term in 1980[1], and the first known use was in 1975[2]. The term connotes that, unlike other tools with only limited uses (such as a hammer, useful for driving nails, but not much else), a computer has many affordances for use and this takes place in an open-ended dialog between the user and the computer.
% Because human–computer interaction studies a human and a machine in conjunction, it draws from supporting knowledge on both the machine and the human side. On the machine side, techniques in computer graphics, operating systems, programming languages, and development environments are relevant. On the human side, communication theory, graphic and industrial design disciplines, linguistics, social sciences, cognitive psychology, and human factors such as computer user satisfaction are relevant. Engineering and design methods are also relevant. Due to the multidisciplinary nature of HCI, people with different backgrounds contribute to its success. HCI is also sometimes referred to as man–machine interaction (MMI) or computer–human interaction (CHI).

% xmodmap i vim http://www.8t8.us/vim/vim.html

% tselect pattern http://stackoverflow.com/questions/4963019/search-for-a-tag

% unite-vim http://www.vim.org/scripts/script.php?script_id=3396

% https://github.com/hernad/neocomplcache

% http://en.wikipedia.org/wiki/Command-line_interface

%A command-line interface (CLI) is a means of interaction with a computer program where the user (or client) issues commands to a program in the form of successive lines of text (command lines).
%The command-line interface evolved from a form of dialog once conducted by humans over teleprinter machines, in which human operators remotely exchanged information, usually one line of text at a time. Early computer systems often used teleprinter machines as the means of interaction with a human operator. The computer became one end of the human-to-human teleprinter model. So instead of a human communicating with another human over a teleprinter, a human communicated with a computer.
%In time, the actual mechanical teleprinter was replaced by a glass tty (keyboard and screen, but emulating the teleprinter), and then by a terminal (where the computer software could address all of the screen, rather than only print successive lines).

% http://www.cs.man.ac.uk/~seanb/teaching/COMP10092/COMP10092-HCI.pdf

Advantages

quick and powerful for experienced users
user-controlled interaction
minimal amount of typing (no mouse use)
can be used in conjunction with other user interfaces

Disadvantages

little or no prompting
requires user nowledge of system, programs
relies on recall of commands and syntax
difficult to learn
error prone  


Menu:

Advantages
Users don't have to memorize complex commands
Structured navigation benefits novices and casual users
Can shorten user learning time and effort
Supports recognition memory


Disadvantages
 May not be appropriate or efficient for some users and tasks
Can force user through many levels of menus
Users may get lost in menu hierarchies
Menu terms and names may not be meaningful to users
Use of modes forces users to follow the system's path



\section{Cloud i command line intefejs}

% http://www.h-online.com/open/news/item/Amazon-releases-preview-of-command-line-for-cloud-services-1774374.html



\section{Zašto developer voli ''vim'', ''tmux'' i ''konzolu''}

\section{Zašto se pojavio PowerShell ?}

Jer su to ''Windows'' developeri tražli.

Zašto su to tražili ?

Radi automatizacije. GUI alati jesu lagani i intuitivni za rukovanje ali se te operacije ne mogu automatizirati.

Koliko god konzola izgledala anahrona u odnosu na GUI interfejs, oni su uočili šta se na toj konzoli, sa bash i sličnim skriptnim jezicima može postići.

\section{REPL}

\subsection irb

\begin{lstlisting}
$ irb
1.9.3p286 :001 > 228 % 15 # cjelobrojni ostatak
 => 3 
1.9.3p286 :002 > 2**3 # stepenovanje
 => 8 
1.9.3p286 :003 > a = 2**3 + 528/5.2
 => 109.53846153846153 
1.9.3p286 :004 > b = Math.sin(0.2) + 2 * Math.cos(0.8)
 => 1.5920827494893923 
1.9.3p286 :005 > a + 2.2 / b
 => 110.92029926059348
 \end{lstlisting}


\section{Zašto najnovija verzija titanium okruženja sadrži moćan CLI ?}

Jer su to developeri tražili

\section{Zašto nijedan git GUI \emph{shell} obezbjeđuje samo osnovni set komandi}

Git je iznimno fleksibilan, ali i kompleksan alat. GUI alati obezbjeđuju obične, najčešće korištene operacije
\begin{itemize}
 \item init, add
 \item pull
 \item commit
 \item push
 \item log, diff
\end{itemize}

Međutim, pored toga


\begin{lstlisting}
~$ git --help

=>

usage: git [--version] [--exec-path[=<path>]] [--html-path] [--man-path] [--info-path]
           [-p|--paginate|--no-pager] [--no-replace-objects] [--bare]
           [--git-dir=<path>] [--work-tree=<path>] [--namespace=<name>]
           [-c name=value] [--help]
           <command> [<args>]

The most commonly used git commands are:
   add        Add file contents to the index
   bisect     Find by binary search the change that introduced a bug
   branch     List, create, or delete branches
   checkout   Checkout a branch or paths to the working tree
   clone      Clone a repository into a new directory
   commit     Record changes to the repository
   diff       Show changes between commits, commit and working tree, etc
   fetch      Download objects and refs from another repository
   grep       Print lines matching a pattern
   init       Create an empty git repository or reinitialize an existing one
   log        Show commit logs
   merge      Join two or more development histories together
   mv         Move or rename a file, a directory, or a symlink
   pull       Fetch from and merge with another repository or a local branch
   push       Update remote refs along with associated objects
   rebase     Forward-port local commits to the updated upstream head
   reset      Reset current HEAD to the specified state
   rm         Remove files from the working tree and from the index
   show       Show various types of objects
   status     Show the working tree status
   tag        Create, list, delete or verify a tag object signed with GPG

See 'git help <command>' for more information on a specific command.
\end{lstlisting}

Međutim, operacije kao što su ''branch'' i ''tag'', ''merge'' se uobičajeno obavljaju sa komandne linije.
One imaju toliko varijanti i mogućih scenarija realizacije, da je implementacija kroz GUI shell gotovo nemoguća. Čak i kada bi se sve to realiziralo, pitanje konačnog korisničkog iskustva takve implementacije je krajnje diskutabilno.

''GUI''


iranje i tagiranje


\begin{lstlisting}
~$ ps ax | grep tmux

=>

75447   ??  Ss     0:49.87 tmux
76174 s007  S+     0:00.01 tmux a
 3148 s014  S+     0:00.00 grep tmux
\end{lstlisting}

\begin{lstlisting}
~$ ps ax | grep tmux | grep -v grep

=>

75447   ??  Rs     0:49.88 tmux
76174 s007  S+     0:00.01 tmux a
\end{lstlisting}

\begin{lstlisting}
~$ ps ax | grep tmux | grep -v grep | grep -c tmux

=> 2
\end{lstlisting}


Zašto db administrator pored svih ''visual'' alata voli sql konzolu ?


\chapter{Zaključak}

Produktivnost


% -------------------------------------------------
%\bibliography{literatura}
%\bibliographystyle{fit}

\end{document}
